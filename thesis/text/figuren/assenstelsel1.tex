\documentclass{article}
\usepackage{tikz}   %TikZ is required for this to work.  Make sure this exists before the next line

\usepackage{tikz-3dplot} %requires 3dplot.sty to be in same directory, or in your LaTeX installation

\usepackage[active,tightpage]{preview}  %generates a tightly fitting border around the work
\PreviewEnvironment{tikzpicture}
\setlength\PreviewBorder{2mm}

\begin{document}

%Angle Definitions
%-----------------

%set the plot display orientation
%synatax: \tdplotsetdisplay{\theta_d}{\phi_d}
\tdplotsetmaincoords{60}{120}

%define polar coordinates for some vector
%TODO: look into using 3d spherical coordinate system
\pgfmathsetmacro{\rvec}{.8}
\pgfmathsetmacro{\thetavec}{50}
\pgfmathsetmacro{\phivec}{0}

%start tikz picture, and use the tdplot_main_coords style to implement the display 
%coordinate transformation provided by 3dplot
\begin{tikzpicture}[scale=5,tdplot_main_coords]

%set up some coordinates 
%-----------------------
\coordinate (O) at (0,0,0);

\filldraw[draw=red,fill=red!20]
	          (-1,0,0)
            -- (1,0,0)
            -- (1,0,1)
            -- (-1,0,1) node[anchor=north east, yshift=-0.8cm,xshift=-0.4cm,rotate=40]{$\varphi=0$}
            --cycle;

%determine a coordinate (P) using (r,\theta,\phi) coordinates.  This command
%also determines (Pxy), (Pxz), and (Pyz): the xy-, xz-, and yz-projections
%of the point (P).
%syntax: \tdplotsetcoord{Coordinate name without parentheses}{r}{\theta}{\phi}
\tdplotsetcoord{P}{\rvec}{\thetavec}{\phivec}

%draw figure contents
%--------------------

%draw the main coordinate system axes
\draw[thick,->] (0,0,0) -- (1,0,0) node[anchor=north east]{$x$};
\draw[thick,->] (0,0,0) -- (0,1,0) node[anchor=north west]{$y$};
\draw[thick,->] (0,0,0) -- (0,0,1) node[anchor=south]{$z$};

%draw a vector from origin to point (P) 
\draw[-stealth,color=blue] (O) -- (P)node[anchor=south,color=blue]{$\vec{P}_{12}$};
\draw[-stealth,color=blue] (O) -- (0,0,0.7) node[anchor=south east]{$\vec{k}_{12}$};


%draw projection on xy plane, and a connecting line


%draw the angle \phi, and label it
%syntax: \tdplotdrawarc[coordinate frame, draw options]{center point}{r}{angle}{label options}{label}
%\tdplotdrawarc{(O)}{0.2}{0}{\phivec}{anchor=north,color=black}{$\phi=0$}


%set the rotated coordinate system so the x'-y' plane lies within the
%"theta plane" of the main coordinate system
%syntax: \tdplotsetthetaplanecoords{\phi}
\tdplotsetthetaplanecoords{\phivec}

%draw theta arc and label, using rotated coordinate system
\tdplotdrawarc[tdplot_rotated_coords]{(0,0,0)}{0.5}{0}{\thetavec}{anchor=north,xshift=-0.25cm,yshift=0.4cm}{$\theta_{kP}$}



%set the rotated coordinate definition within display using a translation
%coordinate and Euler angles in the "z(\alpha)y(\beta)z(\gamma)" euler rotation convention
%syntax: \tdplotsetrotatedcoords{\alpha}{\beta}{\gamma}
\tdplotsetrotatedcoords{\phivec}{\thetavec}{0}

%translate the rotated coordinate system
%syntax: \tdplotsetrotatedcoordsorigin{point}
\tdplotsetrotatedcoordsorigin{(P)}





\end{tikzpicture}

\end{document}
