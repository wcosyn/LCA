\documentclass[12pt]{article}
\usepackage{braket}
\usepackage{amsmath}
\usepackage{amssymb}
\setlength\parindent{0pt}
\usepackage[a4paper, total={7.6in, 10in}]{geometry}
\title{Nuclear Momentum Distributions}
\author{Jarrick Nys}

\begin{document}
\maketitle
\section{Definitions}

\subsection{One-particle momentum distribution}

The One-particle momentum distribution gives the chance of finding a particle with a momentum in the interval $[\vec{k},\vec{k}+d\vec{k}]$. It is given by the following expression
\begin{equation} \label{eq:one_patricle_distr}
	n_1(\vec{k})=\frac{1}{(2\pi)^3}\int d\vec{r}_1 \int d\vec{r}_1' e^{i\vec{k}\cdot (\vec{r}_1-\vec{r}'_1)}\rho_1(\vec{r}_1,\vec{r}_1')
\end{equation}

where $\rho_1(\vec{r}_1,\vec{r}'_1)$ is the one-body non-diagonal density matrix


\begin{equation}
\rho_1(\vec{r}_1,\vec{r}'_1) = \int \{d\vec{r}_{2-A}\} \Psi^*_A(\vec{r}_1,\vec{r}_2,\vec{r}_3, ... ,\vec{r}_A)\Psi_A(\vec{r}_1',\vec{r}_2,\vec{r}_3, ... ,\vec{r}_A).
\end{equation}



Here, $\Psi_A(\vec{r}_1,\vec{r}_2,\vec{r}_3, ... ,\vec{r}_A)$ is the ground state wave function of the nucleus A and with the notation 

\begin{equation}
\{d\vec{r}_{i-A}\}  = d\vec{r}_i d\vec{r}_{i+1}...\vec{r}_A.
\end{equation}
 


For $\Braket{\Psi_A|\Psi_A}=1$, one has that


\begin{equation}
\int d\vec{k}n_1(\vec{k})=1
\end{equation}

In the second quantization formalism one can express the one-particle momentum distribution as

\begin{equation}
n_1(\vec{k})= \bra{\Psi_A} c^\dagger_k c_k \ket{\Psi_A}.
\end{equation}

For the one ony-body non-diagonal density matrix one can write
\begin{align}
\Braket{\vec{r}'_1, \vec{r}'_2, \dotso, \vec{r}'_A|\hat{\rho}|\vec{r}_1, \vec{r}_2, \dotso, \vec{r}_A} = \left( \prod_{\substack{j=1 \\ j \neq i }}^{A} \delta(\vec{r}_j-\vec{r}_j') \right) \sum_{i=1}^{A} \rho(\vec{r}_i)
\end{align}

\begin{align}
\Braket{\Psi_A|\hat{\rho}|\Psi_A} & =\sum_{i=1}^{A} \int d\vec{r}_1 \cdots d\vec{r}_i \cdots d\vec{r}_A \int d\vec{r}'_1 \cdots d\vec{r}'_i \cdots d\vec{r}'_A \Braket{\Psi_A|\vec{r}_1,..., \vec{r}'_i, ..., \vec{r}_A}  \rho(\vec{r}_i, \vec{r}'_i)\Braket{\vec{r}_1,..., \vec{r}_i, ..., \vec{r}_A|\Psi_A} \\
& = \frac{A}{A!} \int d\vec{r}_1 \cdots d\vec{r}_A \int d\vec{r}'_1 \Braket{\Psi_A|\psi^\dagger(\vec{r}'_1) \psi^\dagger(\vec{r}_2) ...\psi^\dagger(\vec{r}_A) \rho(\vec{r}_1,\vec{r}'_1) \psi(\vec{r}_1) ... \psi(\vec{r}_A)|\Psi_A } \\
& = \int d\vec{r} \int d\vec{r}' \Braket{\Psi_A|\psi^\dagger(\vec{r'}) \rho(\vec{r}, \vec{r}') \psi(\vec{r})|\Psi_A } 
\end{align}
So the one-body non-diagonal operator in second quantization becomes
\begin{equation}
\hat{\rho}_{off-diag} = \int d\vec{r} d \vec{r}' \psi^\dagger(\vec{r}) \rho(\vec{r}, \vec{r}') \psi(\vec{r}') 
\end{equation}
The density operator in the second quantization formalism can also be written as
\begin{align}
\rho(\vec{r}_1,\vec{r}'_1) & =  \int \{d\vec{r}_{2-A}\} \Psi^*_A(\vec{r}_1,\vec{r}_2,\vec{r}_3, ...,\vec{r}_A)\Psi_A(\vec{r}_1',\vec{r}_2,\vec{r}_3, ... ,\vec{r}_A) \\
& = \int \{d\vec{r}_{2-A}\} \Braket{\Psi_A|\vec{r}'_1,\vec{r}_2, ...,\vec{r}_A} \Braket{\vec{r}_1,\vec{r}_2, ...,\vec{r}_A |\Psi_A} \\
& = \frac{1}{A!} \int \{d\vec{r}_{2-A}\} \Braket{\Psi_A|\psi^\dagger(\vec{r}'_1) \psi^\dagger(\vec{r}_2) ...\psi^\dagger(\vec{r}_A) \psi(\vec{r}_1) ... \psi(\vec{r}_A)|\Psi_A } \\
& = \Braket{\Psi_A|\psi^\dagger(\vec{r}'_1) \psi(\vec{r}_1)|\Psi_A } \\
& = \Braket{\Psi_A|\sum_{\alpha \beta} c^\dagger_{\alpha} u_\alpha(\vec{r}')c_{\beta} u_\beta(\vec{r})) |\Psi_A }
\end{align}




\subsection{Two-particle momentum distribution}

The two-particle momentum distribution gives the chance of finding a particle with momentum in the interval $[\vec{k}_1,\vec{k}_1+d\vec{k}]$ when there is another particle with a momentum in the interval $[\vec{k}_2,\vec{k}_2+d\vec{k}]$. It is given by the following expression

\begin{equation}
n(\vec{k}_1,\vec{k}_2)=\frac{1}{(2\pi)^6}\int d\vec{r}_1 \int d\vec{r}_2 \int  
    						d\vec{r}_1' \int d\vec{r}_2' 
    						\mathrm{e}^{i\vec{k}_1\cdot (\vec{r}_1-\vec{r}'_1)} 
    						\mathrm{e}^{i\vec{k}_2\cdot(\vec{r}_2-\vec{r}'_2)}
    						\rho_2(\vec{r}_1,\vec{r}_2; \vec{r}_1',\vec{r}_2')
\end{equation}

where $\rho_2(\vec{r}_1,\vec{r}_2, \vec{r}_1',\vec{r}_2')$ is the two-body non-diagonal density matrix


\begin{equation}
\rho_2(\vec{r}_1,\vec{r}_2, \vec{r}_1',\vec{r}_2') = \int \{d\vec{r}_{3-A}\} \Psi^*_A(\vec{r}_1,\vec{r}_2,\vec{r}_3, ... ,\vec{r}_A)\Psi_A(\vec{r}_1',\vec{r}_2',\vec{r}_3, ... ,\vec{r}_A).
\end{equation}

One can also define the two-particle momentum distribution in the relative and centre of mass (rcm) coordinates instead of the centre well (cw) coordinates
\begin{equation}
\vec{r}_{12}= \frac{1}{\sqrt{2}} \left(\vec{r}_1 - \vec{r}_2\right)  
\end{equation}

\begin{equation}
\vec{R}_{12}= \frac{1}{\sqrt{2}} \left(\vec{r}_1 + \vec{r}_2\right)
\end{equation}

\begin{equation}
\vec{p}= \frac{1}{\sqrt{2}} \left(\vec{k}_1 - \vec{k}_2\right)
\end{equation}

\begin{equation}
\vec{P}= \frac{1}{\sqrt{2}} \left(\vec{k}_1 + \vec{k}_2\right)
\end{equation}


\begin{equation}
n(\vec{p},\vec{P})=\frac{1}{(2\pi)^6}
						\int d\vec{r}_{12} \int d\vec{R}_{12} \int d\vec{r}_{12}' \int d\vec{R}_{12}' 
    						\mathrm{e}^{i\vec{p}\cdot (\vec{r}_{12}-\vec{r}_{12}')} 
    						\mathrm{e}^{i\vec{P}\cdot(\vec{R}_{12}-\vec{R}_{12}')} 
    						\rho_2(\vec{r}_{12},\vec{R}_{12}; \vec{r}_{12}',\vec{R}_{12}')
\end{equation}

where 

\begin{equation} \label{eq:twobodydensity}
\rho_2(\vec{r}_{12},\vec{R}_{12}; \vec{r}_{12}',\vec{R}_{12}') = 
							\rho_2\left(	
							\vec{r}_1=\frac{\vec{r}_{12} + \vec{R}_{12}}{\sqrt{2}},
							\vec{r}_2=\frac{-\vec{r}_{12} + \vec{R}_{12}}{\sqrt{2}},
						    \vec{r}_1'=\frac{\vec{r}'_{12} + \vec{R}'_{12}}{\sqrt{2}},	
						    \vec{r}_2'=\frac{-\vec{r}_{12}' + \vec{R}_{12}'}{\sqrt{2}}
						    \right)
\end{equation}

In the second quantization formalism one can write the two-patricle momemtum distribution as

\begin{equation}
n_2(\vec{k_1},\vec{k_2})= \bra{\Psi_A} c^\dagger_{k_1} c^\dagger_{k_2} c_{k_1} c_{k_2} \ket{\Psi_A}
\end{equation}
For the non-diagonal two-body density operator
\begin{equation}
\hat{\rho} = \sum_{i<j} \hat{\rho}(\vec{r}_i, \vec{r}_j; \vec{r}_i', \vec{r}_j')
\end{equation}
with the corresponding matrix element between two states in position space
\begin{equation}
\Braket{\vec{r}'_1, \vec{r}'_2, \dotso, \vec{r}'_A|\hat{\rho}|\vec{r}_1, \vec{r}_2, \dotso, \vec{r}_A} = \left( \prod_{\substack{k\neq i  \\  k\neq j}}^{A} \delta(\vec{r}_k-\vec{r}_k') \right) \sum_{i<j} \hat{\rho}(\vec{r}_i, \vec{r}_j; \vec{r}_i', \vec{r}_j')
\end{equation}
\begin{align}
\rho(\vec{r}_1, \vec{r}_2; \vec{r}_1', \vec{r}_2') & =  \int \{d\vec{r}_{3-A}\} \Psi^*_A(\vec{r}_1,\vec{r}_2,\vec{r}_3, ... ,\vec{r}_A)\Psi_A(\vec{r}_1',\vec{r}_2',\vec{r}_3, ... ,\vec{r}_A) \\
& = \frac{1}{A!} \int \{d\vec{r}_{3-A}\} \Braket{\Psi_A | \psi^\dagger(\vec{r}'_1) \psi^\dagger(\vec{r}'_2)\psi^\dagger(\vec{r}_3) ...\psi^\dagger(\vec{r}_A) \psi(\vec{r}_A) ...\psi(\vec{r}_3)\psi(\vec{r}_2) \psi(\vec{r}_1)|\Psi_A } \\
& = \frac{1}{A(A-1)} \Braket{\Psi_A | \psi^\dagger(\vec{r}'_1) \psi^\dagger(\vec{r}'_2)\psi(\vec{r}_2) \psi(\vec{r}_1)|\Psi_A } \\
& = \frac{1}{A(A-1)} \Braket{\Psi_A | \sum_{\alpha\beta\gamma\delta} c^\dagger_\alpha c^\dagger_\beta u*_\alpha(\vec{r}_1') u*_\beta(\vec{r}_2') u_\gamma(\vec{r}_1) u_\delta(\vec{r}_2)c_\gamma c_\delta|\Psi_A } 
\end{align}





\section{Momentum distributions for IPM}
\subsection{General properties}

In an independent particle model the total wave function of the nucleus is a slater determinant of the one-particle wave functions. A nucleon moves independent in a sort of mean field potential created by all the other nucleons. 

\begin{equation} \label{eq:slater}
\Psi_A(\vec{r}_1,\vec{r}_2,\vec{r}_3, ... ,\vec{r}_A)= \frac{1}{\sqrt{A!}} \sum_{\substack{P}} 
													  (-1)^P \phi_{P_1}(\vec{r}_1)
													         \phi_{P_2}(\vec{r}_2)...
													         \phi_{P_A}(\vec{r}_A)
\end{equation}

where the sum is over all permutations of the indices of the one-particle wave functions. We also have 

\begin{equation} \label{eq:orthogonality}
\int d\vec{r}_i \phi^*_l(\vec{r}_i)\phi_m(\vec{r}_i) = \delta_{lm}
\end{equation}

The one-particle non-diagonal density matrix becomes

\begin{align}
\rho_1(\vec{r}_1,\vec{r}'_1) & = \frac{1}{A!} 	 \sum_{\substack{P}} \sum_{\substack{L}} (-1)^{P+L} \int d\vec{r}_2 d\vec{r}_3 ... d\vec{r}_A 
\phi^*_{P_1}(\vec{r}_1)\phi^*_{P_2}(\vec{r}_2)...\phi^*_{P_A}(\vec{r}_A)
\phi_{L_1}(\vec{r}_1')\phi_{L_2}(\vec{r}_2)...\phi_{L_A}(\vec{r}_A) \\
&  = \frac{1}{A!} 	 \sum_{\substack{P}} \sum_{\substack{L}} (-1)^{P+L} \phi^*_{P_1}(\vec{r}_1) \phi_{L_1}(\vec{r}_1') 
\delta_{P_2,L_2}\delta_{P_3,L_3}...\delta_{P_A,L_A} \\
& = \sum_{\substack{i}} \phi^*_i(\vec{r}_1) \phi_i(\vec{r}_1').
\end{align}

We can plug this into (\ref{eq:one_patricle_distr})

\begin{align} 
	n_1(\vec{k})&=\frac{1}{(2\pi)^3} \sum_{\substack{i}} \int d\vec{r}_1 \int d\vec{r}_1' e^{i\vec{k}\cdot (\vec{r}_1-\vec{r}'_1)}
	\phi^*_i(\vec{r}_1)\phi_i(\vec{r}_1') \\
	& = \sum_{\substack{i}} \tilde{\phi}^*_i(\vec{k})\tilde{\phi}_i(-\vec{k}).
\end{align}

To find an expression for the two-body non-diagonal density one can plug the slater deteminant (\ref{eq:slater}) into equation (\ref{eq:twobodydensity}). Taking into account the orthogonality relation (\ref{eq:orthogonality}) one has

\begin{equation}
\rho_2(\vec{r}_1,\vec{r}_2, \vec{r}_1',\vec{r}_2') = \frac{1}{A(A-1)} \sum_{i \neq j } \phi^*_i(\vec{r}_1) \phi^*_i(\vec{r}_2) \phi_j(\vec{r}_1') \phi_j(\vec{r}_2').
\end{equation}
	
\subsection{IPM for harmonic oscillator potential}
We consider the nucleons moving independently in a spherical symmetric harmonic oscillator potential. From the above we know that we only need to calculate the one-particle wave functions and their fourier transforms. The 3D time independent Schrodinger equation for one patricle is
\begin{equation} \label{eq:HO}
\left( -\frac{\hbar^2}{2M_N} \nabla^2 + \frac{1}{2} M_N \omega^2 r^2 \right) \phi_{nlm}(\vec{r}) = E\phi_{nlm}(\vec{r})
\end{equation}
where the parameter $\hbar\omega$ can be parameterized as
\begin{equation}
\hbar\omega (MeV) = 45A^{-1/3}-25A^{-2/3},
\end{equation}
where A is the mass number of the nucleus. The general solution of (\ref{eq:HO}) is given by
\begin{equation}
\phi_{nlm}(\vec{r}) \equiv \Braket{\vec{r}|nlm} = R_{nl}(r)Y_{lm}(\Omega)	
\end{equation}
where $Y_{lm}(\Omega)$ are the spherical harmonics and the radial wave functions are given in function of the generalized Laguerre polynomals $L^\alpha_n(r)$ by
\begin{equation}
 R_{nl}(r) = \left[ \frac{2n!}{\Gamma(n+l+\frac{3}{2})}\nu^{l+\frac{3}{2}} \right]^{\frac{1}{2}} r^l e^{-\frac{\nu r^2}{2}} L^{l+\frac{1}{2}}_n(\nu r^2)
\end{equation}
where 
\begin{equation}
\nu \equiv \frac{M_N \omega}{\hbar}
\end{equation}
One can calculate the fourier transform of these wave functions explicitly or one can transform equation \ref{eq:HO}, which is written in configuration space, into momentum space
\begin{equation} \label{eq:HO_momentum}
\left( -\frac{M_N \omega^2 \hbar^2}{2} \nabla^2 + \frac{1}{2M_N} k^2 \right) \tilde{\phi}_{nlm}(\vec{k}) = E\tilde{\phi}_{nlm}(\vec{k}).
\end{equation}
One can now see that this equation has the same form as equation (\ref{eq:HO}). So the solutions have the same form
\begin{equation}
\phi_{nlm}(\vec{k}) \equiv \Braket{\vec{k}|nlm} = K_{nl}(k)Y_{lm}(\Omega)	
\end{equation}
where $Y_{lm}(\Omega)$ are the spherical harmonics and the radial wave functions are given in function of the generalized Laguerre polynomals $L^\alpha_n(k)$ by
\begin{equation}
 K_{nl}(k) = \left[ \frac{2n!}{\Gamma(n+l+\frac{3}{2})}\nu'^{l+\frac{3}{2}} \right]^{\frac{1}{2}} k^l e^{-\frac{\nu' k^2}{2}} L^{l+\frac{1}{2}}_n(\nu' r^2)
\end{equation}
where 
\begin{equation}
\nu' \equiv \frac{\hbar}{M_N \omega}
\end{equation}

\end{document}

